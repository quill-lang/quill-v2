\section{Inductive families}
We present a system of inductive types similar to that described by Dybjer in \cite{Dybjer1994}.
\begin{defn}
	An \textit{inductive family} is composed of
	\begin{itemize}
		\item a qualified name \( q \);
		\item a list of universe parameters \( U \in \mathcal N \ast \);
		\item a type \( I \), which is to be the type of the inductive type to be declared;
		\item a list of intro rules \( \mathsf{intro_n} : i_n \),
	\end{itemize}
	subject to various conditions laid out in the remainder of this section.
\end{defn}
The type \( I \) must be of the form
\[ I = \prd{\vb A : \bm \sigma}{\vb a : \bm \alpha} \Sort u \]
where \( \vb A, \vb a \) are lists of parameters and \( \bm \sigma, \bm \alpha \) are lists of types.
We call \( \vb A \) the \textit{global parameters} and \( \vb a \) the \textit{index parameters}.
The combined list of parameters is used to index a particular type from a family of types being constructed at once.
When we construct the inductive family, we add the assumption \( q : I \) to the environment \( \Gamma \) (more precisely, we add \( \langle q, U, I \rangle \)), essentially declaring the inductive type available for use in later expressions.

The type of each intro rule must be of the form
\[ i_n = \prd{\vb A : \bm \sigma}{\vb b : \bm \beta}{\vb u : \bm \gamma} I\ \vb A\ (p\ \vb A\ \vb b) \]
where
\begin{itemize}
	\item \( \bm \beta \) and \( \bm \gamma \) are lists of types
	\item each \( \gamma_i \) has the form
	\[ \gamma_i = \prd{\vb x:\bm \xi_i} I\ \vb A\ (p_i\ \vb A\ \vb b\ \vb x) \]
	where \( \bm \xi_i \) is a list of types that cannot contain \( I \) (this is known as the \textit{positivity constraint}), and
	\[ p_i : \prd{\vb A : \bm \sigma}{\vb b : \bm \beta}{\vb x:\bm \xi_i} \bm \alpha \]
	\item \( p \) has the form
	\[ p : \prd{\vb A : \bm \sigma}{\vb b : \bm \beta} \bm \alpha \]
\end{itemize}
We call \( \vb b \) the \textit{nonrecursive arguments} and \( \vb u \) the \textit{recursive arguments}.
In our kernel, we do not enforce the order in which elements \( \vb b \) and \( \vb u \) may occur; the recursive and nonrecursive arguments may be interspersed.

\begin{eg}[natural numbers]
	Let
	\begin{itemize}
		\item \( q = \mathsf{nat} \);
		\item \( U = [] \);
		\item \( I = \Type \), so \( \bm \sigma = \bm \alpha = [] \) and \( u = 1 \);
		\item \( \mathsf{intro}_1 = \mathsf{zero} : \mathsf{nat} \), so \( \bm \beta = \bm \gamma = [] \);
		\item \( \mathsf{intro}_2 = \mathsf{succ} : \prd{n:\mathsf{nat}} \mathsf{nat} \), so \( \bm \beta = [] \) and \( \bm \gamma = [\mathsf{nat}] \).
	\end{itemize}
	This is the definition of the Peano natural numbers.
\end{eg}
\begin{eg}[lists]
	Let
	\begin{itemize}
		\item \( q = \mathsf{list} \);
		\item \( U = [u] \);
		\item \( I = \prd{T:!\Sort (u + 1)} \Sort (u + 1) \), so \( \bm \sigma = [\Sort (u + 1)] \), \( \bm \alpha = [] \), and the universe parameter is \( u + 1 \);
		\item \( \mathsf{intro}_1 = \mathsf{nil} : \prd{T:!\Sort (u + 1)} \mathsf{list}\ T \), so \( \bm \beta = \bm \gamma = [] \);
		\item \( \mathsf{intro}_2 = \mathsf{cons} : \prd{T:!\Sort (u + 1)}{x:T}{t:\mathsf{list}\ T} \mathsf{list}\ T \), so \( \bm\beta = [T] \) and \( \bm\gamma = [\mathsf{list}\ T] \).
	\end{itemize}
\end{eg}
\begin{eg}[arrays]
	Let
	\begin{itemize}
		\item \( q = \mathsf{array} \);
		\item \( U = [u] \);
		\item \( I = \prd{T:!\Sort (u + 1)}{n:!\mathsf{nat}} \Sort (u + 1) \), so \( \bm \sigma = [\Sort (u + 1)] \), \( \bm \alpha = [\mathsf{nat}] \), and the universe parameter is \( u + 1 \);
		\item \( \mathsf{intro}_1 = \mathsf{anil} : \prd{T:!\Sort (u + 1)} \mathsf{array}\ T\ \mathsf{zero} \), so \( \bm \beta = \bm \gamma = [] \);
		\item \( \mathsf{intro}_2 = \mathsf{acons} : \prd{T:!\Sort (u + 1)}{n:!\mathsf{nat}}{x:T}{t:\mathsf{array}\ T\ n} \mathsf{array}\ T\ (\mathsf{succ}\ n) \), so \( \bm\beta = [\mathsf{nat}, T] \) and \( \bm\gamma = [\mathsf{array}\ T\ n] \).
	\end{itemize}
\end{eg}
\begin{eg}[equality]
	Let
	\begin{itemize}
		\item \( q = \mathsf{eq} \);
		\item \( U = [u] \);
		\item \( I = \prd{T:!\Sort u}{x:T}{y:T} \Prop \), so \( \bm \sigma = [\Sort u, T] \) and \( \bm \alpha = [T] \), and the universe parameter is zero;
		\item \( \mathsf{intro}_1 = \mathsf{refl} : \prd{T:!\Sort u}{x:T}\mathsf{eq}\ T\ x\ x \), so \( \bm \beta = \bm \gamma = [] \).
	\end{itemize}
\end{eg}
\begin{eg}[\( \mathsf{W} \)-types: well-founded trees]
	Let
	\begin{itemize}
		\item \( q = \mathsf{W} \);
		\item \( U = [u, v] \);
		\item \( I = \prd{S:!\Sort (u + 1)}{T:!S \to \Sort (v + 1)} \Sort (\max\ u\ v + 1) \), so \( \bm \sigma = [\Sort (u + 1), S \to \Sort (v + 1)] \) and \( \bm \alpha = [] \), and the universe parameter is \( \max\ u\ v + 1 \), noting that \( S \to \Sort (v + 1) \) is shorthand for \( \prd{s:S} \Sort(v+1) \);
		\item \( \mathsf{intro}_1 = \mathsf{W.mk} : \prd{S:!\Sort (u + 1)}{T:!S \to \Sort (v + 1)}{x:S}{y:\prd{z:T\ x} \mathsf{W}\ S\ T} \mathsf{W}\ S\ T \), so \( \bm \beta = [S] \) and \( \bm \gamma = \qty[\prd{z:T\ x}\mathsf{W}\ S\ T] \).
	\end{itemize}
\end{eg}

\section{Elimination rule}
Given the above, we construct the elimination rule
\[ \mathsf{rec} : \prd{\vb A : \bm \sigma}{C : \prd{\vb a : \bm \alpha}{c : I\ \vb A\ \vb a} \Sort \ell}{\vb e : \bm \epsilon}{\vb a : \bm \alpha}{c : I\ \vb A\ \vb a} C\ \vb a\ c \]
\( C \) is called the \textit{type former} or \textit{motive} for the recursion.
\( \ell \) is a universe parameter to the recursor, constrained by the following rules.
\begin{itemize}
	\item If \( u \) is never the zero universe, \( \ell \) is arbitrary.
	\item Otherwise, \begin{itemize}
		\item If there are zero introduction rules, \( \ell \) is arbitrary.
		\item If there is exactly one introduction rule, then
		\begin{itemize}
			\item If each argument to the intro rule that is not in \( \vb b \) or \( \vb u \) either lives in \( \Prop \) or occurs in the return type, \( \ell \) is arbitrary.
			\item If there is any argument that does not satisfy those conditions, \( \ell \) must only take the value 0.
		\end{itemize}
		\item If there is more than one introduction rule, \( \ell \) must only take the value 0.
	\end{itemize}
\end{itemize}
For each intro rule \( i_n \), we have a \textit{minor premise} \( e_n \), of type
\[ \epsilon_n = \prd{\vb b : \bm \beta}{\vb u : \bm \gamma}{\vb v : \bm \delta} C\ (p\ \vb A\ \vb b)\ (i_n\ \vb A\ \vb b\ \vb u) \]
where \( \bm \delta \) has the same length as \( \bm \gamma \), and
\[ \delta_i = \prd{\vb x:\bm \xi_i} C\ (p_i\ \vb A\ \vb b\ \vb x)\ (u_i\ \vb x) \]
The computation rules for the recursor are of the form
\[ \inferrule*[right=$\mathsf{rec}$-\rcomp]
	{\oftp\Gamma{\vb A}{\bm \sigma} \and \oftp\Gamma{C}{\prd{\vb a : \bm \alpha}{c : I\ \vb A\ \vb a} \Sort \ell} \and \oftp\Gamma{\vb e}{\bm \epsilon} \and \oftp\Gamma{\vb b}{\bm \beta} \and \oftp\Gamma{\vb u}{\bm \gamma}}
	{\Gamma\vdash {\mathsf{rec}\ \vb A\ C\ \vb e\ (p\ \vb A\ \vb b)\ (\mathsf{intro}_n\ \vb A\ \vb b\ \vb u)}\jdeq{e_i\ \vb b\ \vb u\ \vb v}} \]
given that the types \( \bm \sigma, \bm \epsilon, \bm \beta, \bm \gamma \) and the universe \( \ell \) satisfy the requirements above.
This yields the \( \iota \)-reduction rule
\[ \inferrule
	{\oftp\Gamma{\vb A}{\bm \sigma} \and \oftp\Gamma{C}{\prd{\vb a : \bm \alpha}{c : I\ \vb A\ \vb a} \Sort \ell} \and \oftp\Gamma{\vb e}{\bm \epsilon} \and \oftp\Gamma{\vb b}{\bm \beta} \and \oftp\Gamma{\vb u}{\bm \gamma}}
	{{\mathsf{rec}\ \vb A\ C\ \vb e\ (p\ \vb A\ \vb b)\ (\mathsf{intro}_n\ \vb A\ \vb b\ \vb u)}\;\triangleright_\iota\;{e_i\ \vb b\ \vb u\ \vb v}} \]

\begin{eg}[natural numbers]
	The elimination rule for natural numbers is
	\[ \rec{\mathsf{nat}} : \prd{C:\prd{c:\mathsf{nat}} \Sort \ell}{\vb e : \bm \epsilon}{c : \mathsf{nat}} C\ c \]
	where
	\begin{align*}
		\epsilon_1 = \epsilon_{\mathsf{zero}} &= C\ \mathsf{zero} \\
		\epsilon_2 = \epsilon_{\mathsf{succ}} &= \prd{u:\mathsf{nat}}{v:C\ u} C\ (\mathsf{succ}\ u)
	\end{align*}
	This is precisely the induction principle for natural numbers; the minor premise \( \epsilon_{\mathsf{zero}} \) is the base case of the induction and \( \epsilon_{\mathsf{succ}} \) is the inductive step.
\end{eg}
\begin{eg}[lists]
	The elimination rule for lists (of arbitrary length) is
	\[ \rec{\mathsf{list}} : \prd{T:!\Sort (u+1)}{C:\prd{c:\mathsf{list}\ T} \Sort \ell}{\vb e : \bm \epsilon}{c:\mathsf{list}\ T} C\ c \]
	where
	\begin{align*}
		\epsilon_1 = \epsilon_{\mathsf{nil}} &= C\ (\mathsf{nil}\ T) \\
		\epsilon_2 = \epsilon_{\mathsf{cons}} &= \prd{b:T}{u:\mathsf{list}\ T}{v:C\ u} C\ (\mathsf{cons}\ T\ b\ u)
	\end{align*}
\end{eg}
\begin{eg}[arrays]
	The elimination rule for arrays of length given by an index parameter is
	\[ \rec{\mathsf{array}} : \prd{T:!\Sort (u+1)}{C:\prd{n:!\mathsf{nat}}{c:\mathsf{array}\ T\ n}\Sort \ell}{\vb e : \bm \epsilon}{n:!\mathsf{nat}}{c:\mathsf{array}\ T\ n}C\ n\ c \]
	where
	\begin{align*}
		\epsilon_1 = \epsilon_{\mathsf{anil}} &= C\ \mathsf{zero}\ (\mathsf{anil}\ T) \\
		\epsilon_2 = \epsilon_{\mathsf{acons}} &= \prd{n:!\mathsf{nat}}{b:T}{u:\mathsf{array}\ T\ n}{v:C\ n\ u} C\ (\mathsf{succ}\ n)\ (\mathsf{acons}\ T\ b\ u)
	\end{align*}
	Note that the parameter \( n \) is marked as a compile time expression wherever it appears.
\end{eg}
\begin{eg}[equality]
	The elimination rule for the equality type is
	\[ \rec{\mathsf{eq}} : \prd{T:!\Sort u}{x:T}{C:\prd{y:T}{c:\mathsf{eq}\ T\ x\ y}\Sort \ell}{e : \epsilon}{y:T}{c:\mathsf{eq}\ T\ x\ y}C\ y\ x \]
	where
	\[ \epsilon = \epsilon_{\mathsf{refl}} = C\ x\ (\mathsf{refl}\ T\ x) \]
	Informally, this says that if \( \mathsf{eq}\ T\ x\ y \), it suffices to assume that \( x \) and \( y \) are the same (definitionally equal, from the perspective of the minor premise).
\end{eg}
\begin{eg}[\( \mathsf{W} \)-types]
	The elimination rule for the type of well-founded trees is
	\[ \rec{\mathsf{W}} : \prd{S:!\Sort(u+1)}{T:!S\to\Sort(v+1)}{C:\prd{c:\mathsf{W}\ S\ T}\Sort \ell}{e:\epsilon}{c:\mathsf{W}\ S\ T}C\ c \]
	where
	\[ \epsilon = \prd{x:S}{y:\prd{z:T\ x}\mathsf{W}\ S\ T}{v:\prd{z:T\ x}C\ (y\ z)}C\ (\mathsf{W.mk}\ S\ T\ x\ y) \]
\end{eg}

% should C be marked comptime? no - because we may want to let C be in Prop, still allowed

\section{Squashed types}
To work with \( \Delta \) types of inductives, we create three new gadgets: squashed types, the squash function, and the borrowed form of the recursor.
Each is necessary to fully characterise the behaviour of \( \Delta \) types when the parameter is an inductive.

To start, we define the squashed type \( I^{\mathcal S} \) of an inductive type \( I \) with parameters as above.

\begin{defn}
	We define the function \( \mathcal D_\rho \) on binders \( (b_i : \beta_i) \) such that
	\begin{itemize}
		\item compile-time binders \( (b_i : !\beta_i) \) are mapped to themselves;
		\item if \( \beta_i \) is of the form \( \Delta_{\rho'} \beta_i' \), the binder is mapped to itself (the \( \ast \)-reduction rule allows for this);
		\item all other binders \( (b_i : \beta_i) \) are mapped to \( (b_i : \Delta_{\rho} \beta_i) \).
	\end{itemize}
	We further define the associated function \( \mathcal B_\rho \) on variables \( b_i \) bound in a binder \( (b_i : \beta_i) \) such that
	\begin{itemize}
		\item variables \( b_i \) bound in compile-time binders \( (b_i : !\beta_i) \) are mapped to themselves;
		\item variables \( b_i \) of type \( \beta_i = \Delta_{\rho'} \beta_i' \) are mapped to themselves;
		\item all other variables \( b_i \) are mapped to \( \&_{\rho} b_i \).
	\end{itemize}
	Hence, for a binder \( (b_i : \beta_i) \), the variable \( \mathcal B_\rho\ b_i \) is a valid parameter for the binder \( \mathcal D_\rho\ (b_i : \beta_i) \). These two functions are only used for defining squashed types; they have no place in Feather code itself.
\end{defn}

Let \( \vb b' \) be the subsequence of \( \vb b \) such that \( \mathcal B_\rho b_i = b_i \). That is, \( \vb b' \) is the subsequence such that either the associated binder \( (b_i : \beta_i) \) is marked as compile time: \( (b_i : !\beta_i) \), or \( \beta_i \) is already a borrowed type.
We operate under the restriction that \( p \) and \( p_i \) are functions only of \( \vb A \) and \( \vb b' \); we may depend only on compile time or borrowed nonrecursive parameters.

The squashed type has the following properties:
\begin{itemize}
	\item its name is \( q^{\mathcal S} \), defined as \( q \) but with a fixed suffix to denote that it is the squashed variant;
	\item it has the same universe parameters \( U \);
	\item its type is \( I^{\mathcal S} = \prd{\rho : \Region} I \);
	\item its intro rules \( \mathsf{intro}_n^{\mathcal S} : i_n^{\mathcal S} \) have different types:
\end{itemize}
\[ i_n^{\mathcal S} = \prd{\rho : \Region}{\vb A : \bm \sigma}{\mathcal D_\rho[\vb b : \bm \beta]}{\mathcal D_\rho[\vb u : \bm \gamma]} I^{\mathcal S}\ \rho\ \vb A\ (p\ \vb A\ \vb b') \]
Since the \( \Delta \) map preserves universes, the recursor for the squashed type is almost identical:
\[ \mathsf{rec}^{\mathcal S} : \prd{\rho : \Region}{\vb A : \bm \sigma}{C : \prd{\vb a : \bm \alpha}{c : I^{\mathcal S}\ \rho\ \vb A\ \vb a} \Sort \ell}{\vb e : \bm \epsilon^{\mathcal S}}{\vb a : \bm \alpha}{c : I^{\mathcal S}\ \rho\ \vb A\ \vb a} C\ \vb a\ c \]
where
\[ \epsilon_n^{\mathcal S} = \prd{\mathcal D_\rho[\vb b : \bm \beta]}{\mathcal D_\rho[\vb u : \bm \gamma]} C\ (p\ \vb A\ \vb b')\ (i_n^{\mathcal S}\ \rho\ \vb A\ \vb b\ \vb u) \]
Note that there are no recursive (with respect to \( I^{\mathcal S} \), not \( I \)) arguments in the squashed type, since \( I^{\mathcal S} \) is defined after \( I \).
Its computation rules are defined as above.
We can now define the squash function
\[ \mathsf{squash} : \prd{\rho : \Region}{\vb A : \bm \sigma}{\vb a : \bm \alpha}{c : \Delta_\rho\ I\ \vb A\ \vb a} I^{\mathcal S}\ \rho\ \vb A\ \vb a \]
It has computation rules of the form
\[ \inferrule*[right=$\mathsf{squash}$-\rcomp]
	{\oftp\Gamma{\rho}{\Region} \and \oftp\Gamma{\vb A}{\bm \sigma} \and \oftp\Gamma{\vb b}{\bm \beta} \and \oftp\Gamma{\vb u}{\bm \gamma}}
	{\jdeqtp\Gamma{\mathsf{squash}\ \rho\ \vb A\ \vb a\ (\&_\rho\ (\mathsf{intro}_n\ \vb A\ \vb b\ \vb u))}{\mathsf{intro}_n^{\mathcal S}\ \rho\ \vb A\ \mathcal B_\rho[\vb b]\ \mathcal B_\rho[\vb u]}{I^{\mathcal S}\ \rho\ \vb A\ (p\ \vb A\ \vb b')}} \]
This function can be thought of as propagating the borrow into the elements of a type.
The associated rewrite rule is called \( \mathcal S \)-reduction.
\[ \inferrule
	{\oftp\Gamma{\rho}{\Region} \and \oftp\Gamma{\vb A}{\bm \sigma} \and \oftp\Gamma{\vb b}{\bm \beta} \and \oftp\Gamma{\vb u}{\bm \gamma}}
	{\mathsf{squash}\ \rho\ \vb A\ \vb a\ (\&_\rho\ (\mathsf{intro}_n\ \vb A\ \vb b\ \vb u)) \;\triangleright_{\mathcal S}\;\mathsf{intro}_n^{\mathcal S}\ \rho\ \vb A\ \mathcal B_\rho[\vb b]\ \mathcal B_\rho[\vb u]} \]
Note that since \( p \) is only allowed to depend on those \( b_i \) that were unchanged under the \( \mathcal B_\rho \) map, the squash function is type correct.

\section{Borrowed form of the recursor}
We now create another recursor for each type that has a squash function.
It has type
\[ \mathsf{recb} : \prd{\rho : \Region}{\vb A : \bm \sigma}{C : \prd{\vb a : \bm \alpha}{c : I^{\mathcal S}\ \rho\ \vb A\ \vb a} \Sort \ell}{\vb e : \bm \epsilon'}{\vb a : \bm \alpha}{c : \Delta_\rho\ I\ \vb A\ \vb a} C\ \vb a\ (\mathsf{squash}\ \rho\ \vb A\ \vb a\ c) \]
where
\[ \epsilon_n' = \prd{\mathcal D_\rho[\vb b : \bm \beta]}{\mathcal D_\rho[\vb u : \bm \gamma]}{\vb v : \bm \delta'} C\ (p\ \vb A)\ (i_n^{\mathcal S}\ \vb A\ \vb b\ \vb u) \]
where \( \bm \delta' \) has the same length as \( \bm \gamma \), and
\[ \delta_i' = \prd{\vb x:\bm \xi_i} C\ (p_i\ \vb A\ \vb b'\ \vb x)\ (u_i\ \vb x) \]
The associated computation rules are of the form
\[ \inferrule*[right=$\mathsf{recb}$-\rcomp]
	{\oftp\Gamma{\rho}{\Region} \and \oftp\Gamma{\vb A}{\bm \sigma} \and \oftp\Gamma{C}{\prd{\vb a : \bm \alpha}{c : I^{\mathcal S}\ \rho\ \vb A\ \vb a} \Sort \ell} \\\\ \oftp\Gamma{\vb e}{\bm \epsilon'} \and \oftp\Gamma{\vb b}{\bm \beta} \and \oftp\Gamma{\vb u}{\bm \gamma}}
	{\Gamma\vdash {\mathsf{recb}\ \rho\ \vb A\ C\ \vb e\ (p\ \vb A\ \vb b')\ (\&_\rho\ \mathsf{intro}_n\ \vb A\ \vb b\ \vb u)}\jdeq{e_i\ \mathcal B_\rho[\vb b]\ \mathcal B[\vb u]\ \vb v}} \]
The reduction rule, \( \iota^{\mathcal S} \)-reduction, is
\[ \inferrule*[right=$\mathsf{recb}$-\rcomp]
	{\oftp\Gamma{\rho}{\Region} \and \oftp\Gamma{\vb A}{\bm \sigma} \and \oftp\Gamma{C}{\prd{\vb a : \bm \alpha}{c : I^{\mathcal S}\ \rho\ \vb A\ \vb a} \Sort \ell} \\\\ \oftp\Gamma{\vb e}{\bm \epsilon'} \and \oftp\Gamma{\vb b}{\bm \beta} \and \oftp\Gamma{\vb u}{\bm \gamma}}
	{{\mathsf{recb}\ \rho\ \vb A\ C\ \vb e\ (p\ \vb A\ \vb b')\ (\&_\rho\ \mathsf{intro}_n\ \vb A\ \vb b\ \vb u)} \;\triangleright_{\iota^{\mathcal S}}\; {e_i\ \mathcal B_\rho[\vb b]\ \mathcal B[\vb u]\ \vb v}} \]
